\chapter{Atividades desenvolvidas}

Adicione aqui um parágrafo introdutório para que o capítulo não comece ``vazio'' e a leitura não entre diretamente na primeira seção. Você pode começar o parágrafo falando do assunto e dando continuidade nas seções seguintes ou começar com um parágrafo genérico que descreve o que os leitores encontrarão. Exemplo:

\textit{``Esse capítulo apresentará...''}

\section{Escopo do Projeto}
Defina os limites do projeto, ou seja, o que está incluído no escopo e o que está excluído. Isso ajuda a evitar mal-entendidos futuros.

Liste e explique as funcionalidades e recursos do sistema. Isso pode incluir ações que os usuários podem realizar, como login, busca, envio de informações etc. Apresente aqui o Levantamento de Requisitos contendo os Requisitos Funcionais (RF) e os Não-Funcionais (RNF);

\section{Banco de Dados}
Explique como os dados são armazenados e gerenciados pelo sistema, incluindo detalhes sobre o banco de dados utilizado, esquema de banco de dados, tabelas e relacionamentos. Apresente aqui o Diagrama Entidade-Relacionamento (DER) e o Modelo Entidade-Relacionamento (MER);

\section{Arquitetura do Sistema}
Descreva a arquitetura geral do sistema. Explique como está a estruturação dos arquivos, como os códigos estão interligados e de que forma as funções estão associadas.

Caso deseje, separe a explicação desse tópico em front-end e back-end.

%\subsection{Front-end}
%\subsection{Back-end}

\section{Tecnologias e Ferramentas}
Mencione as tecnologias, linguagens de programação, \textit{frameworks}, bibliotecas (\textit{libs}) e demais ferramentas que foram utilizadas no desenvolvimento do sistema. Apresente aqui imagens e/ou logotipos das ferramentas.
