\chapter{CONSIDERAÇÕES FINAIS}
% ---

Com base nas pesquisas e conversas com alguns trabalhadores que residem em cidades vizinhas, e precisam se deslocarem para a cidade onde ficam localizados o porto de mineração e sedes da SIGMA Lítio, prestadoras terceirizadas para a mesma, tendo uma dificuldade para encontrar pontos comerciais e residenciais para residirem, Onde detectamos a necessidade de realizarmos em esquipe um sistema que todos tenham a facilidade de encontrar casas e pontos comerciais tanto para locação quanto para locatários independentes auxiliando juntamente com: Contratos Redigidos e orientados por um profissional na área de Direitos Imobiliários, sendo encontrados nos seguintes Anexos A e B. Todo contrato de aluguel deve seguir as leis vigentes na área, respeitando e seguindo a Constituição Brasileira. Para embasar nosso aplicativo, consultamos o advogado Dr. David Torres Silva, que redigiu dois modelos de contrato de aluguel.

O tema do trabalho era o desenvolvimento de uma aplicação web para a alocação de imóveis, a fim de facilitar a comunicação entre locador e locatário na cidade de Araçuaí e região, em que fosse possível fazer o cadastro de imóveis, agendamento de visitas, um sistema de chats e um modelo de contrato para que pessoas com pouco acesso a advogados soubessem como funciona um contrato de locação. Houve muitos marcos positivos, como o aprendizado de novas tecnologias, e pontos negativos, como a falta de suporte a bibliotecas antigas em novas versões do Laravel.\\

Acreditamos que o sistema tem muito a evoluir; as evoluções propostas são a criação de um sistema de avaliação, a criação de um sistema de casas favoritas e um sistema de denúncia de usuários/casas.






