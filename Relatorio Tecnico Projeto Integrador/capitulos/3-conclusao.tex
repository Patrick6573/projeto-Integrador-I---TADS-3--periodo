\chapter{Considerações finais}
% ---
Com base nas pesquisas e conversas com alguns trabalhadores que residem em cidades vizinhas, e precisam se deslocarem para a cidade onde ficam localizados o porto de mineração e sedes da SIGMA Lítio, prestadoras terceirizadas para a mesma, tendo uma dificuldade para encontrar pontos comerciais e residenciais para residirem.
Onde detectamos a necessidade de realizarmos em esquipe um sistema que todos tenham a facilidade de encontrar casas e pontos comerciais tanto para locação quanto para locatários independentes auxiliando juntamente com: Contratos Redigidos e orientados por um profissional na área de Direitos Imobiliários, sendo encontrados nos seguintes Anexos A e B. DAVID TORRES ADVOGADOS ASSOCIADOS

%%Faça um breve resumo do tema e dos objetivos propostos para o trabalho. Em seguida, recapitule os principais resultados obtidos durante o desenvolvimento do sistema, apontando os marcos positivos e negativos que ocorreram durante o processo.

%Para finalizar o capítulo, descreva as limitações de sua solução e apresente propostas de melhorias futuras para o sistema.

--------------

%Para ajudá-los a fazer citações de artigos, livros, sites e demais fontes de informação em LaTeX, deixarei esse pequeno parágrafo. Basicamente, há três formas de citar um material: 1) comando \textbf{cite}; 2) comando \textbf{citeonline}; e 3) o bloco \textbf{citacao}. 

%O comando \textbf{cite} apresenta o seguinte formato no PDF: \cite[p. 23]{abntex2-wiki-como-customizar}. O comando \textbf{cite} é usado quando se faz uma \textbf{citação direta curta}, após colocar-se o texto retirado da fonte entre as aspas.

%Já o comando \textbf{citeonline} apresenta o seguinte formato no PDF: \citeonline{abntex2-wiki-como-customizar}. Esse comando é usado quando se faz uma \textbf{citação indireta}, que é aquela onde se replica a \textbf{ideia do autor, mas as palavras são suas}. Esse tipo de formato de citação costuma aparecer ``entre o texto'', não no final.

%Por fim, o bloco citacao é utilizado para citações diretas que possuem mais de 3 linhas. Elas têm formatação diferente conforme as regras da \citeonline{NBR6024:2012}.

%\begin{citacao}
  %  Esta é uma \textbf{citação direta longa}, ou seja, é uma citação que possui mais de 3 linhas copiadas literalmente de alguma fonte de informação utilizada no trabalho. Observe que o bloco \textbf{citacao} é usado em conjunto com o comando \textbf{cite}. \cite[p. 11]{abntex2-wiki-como-customizar}.%
  %\end{citacao}