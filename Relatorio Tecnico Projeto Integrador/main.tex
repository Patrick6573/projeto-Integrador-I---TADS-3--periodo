%% abtex2-modelo-relatorio-tecnico.tex, v-1.9.6 laurocesar
%% Copyright 2012-2016 by abnTeX2 group at http://www.abntex.net.br/ 
%%
%% This work may be distributed and/or modified under the
%% conditions of the LaTeX Project Public License, either version 1.3
%% of this license or (at your option) any later version.
%% The latest version of this license is in
%%   http://www.latex-project.org/lppl.txt
%% and version 1.3 or later is part of all distributions of LaTeX
%% version 2005/12/01 or later.
%%
%% This work has the LPPL maintenance status `maintained'.
%% 
%% The Current Maintainer of this work is the abnTeX2 team, led
%% by Lauro César Araujo. Further information are available on 
%% http://www.abntex.net.br/
%%
%% This work consists of the files abntex2-modelo-relatorio-tecnico.tex,
%% abntex2-modelo-include-comandos and abntex2-modelo-references.bib
%%

% ------------------------------------------------------------------------
% ------------------------------------------------------------------------
% abnTeX2: Modelo de Relatório Técnico/Acadêmico em conformidade com 
% ABNT NBR 10719:2015 Informação e documentação - Relatório técnico e/ou
% científico - Apresentação
% ------------------------------------------------------------------------ 
% ------------------------------------------------------------------------

\documentclass[
	% -- opções da classe memoir --
	12pt,				% tamanho da fonte
	openright,			% capítulos começam em pág ímpar (insere página vazia caso preciso)
	oneside,			% para impressão em recto e verso. Oposto a oneside
	a4paper,			% tamanho do papel. 
	% -- opções da classe abntex2 --
	%chapter=TITLE,		% títulos de capítulos convertidos em letras maiúsculas
	%section=TITLE,		% títulos de seções convertidos em letras maiúsculas
	%subsection=TITLE,	% títulos de subseções convertidos em letras maiúsculas
	%subsubsection=TITLE,% títulos de subsubseções convertidos em letras maiúsculas
	% -- opções do pacote babel --
	english,			% idioma adicional para hifenização
	french,				% idioma adicional para hifenização
	spanish,			% idioma adicional para hifenização
	brazil,				% o último idioma é o principal do documento
	]{abntex2}


% ---
% PACOTES
% ---

% ---
% Pacotes fundamentais 
% ---
\usepackage{lmodern}			% Usa a fonte Latin Modern
\usepackage[T1]{fontenc}		% Selecao de codigos de fonte.
\usepackage[utf8]{inputenc}		% Codificacao do documento (conversão automática dos acentos)
\usepackage{indentfirst}		% Indenta o primeiro parágrafo de cada seção.
\usepackage{color}				% Controle das cores
\usepackage{graphicx}			% Inclusão de gráficos
\usepackage{microtype} 			% para melhorias de justificação
% ---

% ---
% Pacotes adicionais, usados no anexo do modelo de folha de identificação
% ---
\usepackage{multicol}
\usepackage{multirow}
% ---
	
% ---
% Pacotes adicionais, usados apenas no âmbito do Modelo Canônico do abnteX2
% ---
\usepackage{lipsum}				% para geração de dummy text
% ---

% ---
% Pacotes de citações
% ---
\usepackage[brazilian,hyperpageref]{backref}	 % Paginas com as citações na bibl
\usepackage[alf]{abntex2cite}	% Citações padrão ABNT

% --- 
% CONFIGURAÇÕES DE PACOTES
% --- 

% ---
% Configurações do pacote backref
% Usado sem a opção hyperpageref de backref
\renewcommand{\backrefpagesname}{Citado na(s) página(s):~}
% Texto padrão antes do número das páginas
\renewcommand{\backref}{}
% Define os textos da citação
\renewcommand*{\backrefalt}[4]{
	\ifcase #1 %
		Nenhuma citação no texto.%
	\or
		Citado na página #2.%
	\else
		Citado #1 vezes nas páginas #2.%
	\fi}%
% ---

% ---
% Informações de dados para CAPA e FOLHA DE ROSTO
% ---
\titulo{Insira o título de seu trabalho}
\autor{Insira os nomes dos membros (um nome abaixo do outro)}
\local{Araçuaí--MG}
\data{2024}
\instituicao{%
  Instituto Federal de Educação, Ciência e Tecnologia do Norte de Minas Gerais (IFNMG)
  \par
  Tecnologia em Análise e Desenvolvimento de Sistemas
  \par
  Núcleo de Informática}
\tipotrabalho{Relatório técnico}
% O preambulo deve conter o tipo do trabalho, o objetivo, 
% o nome da instituição e a área de concentração 
\preambulo{Relatório técnico para descrição da modelagem, codificação e demais atividades realizadas durante o Projeto Integrador.}
% ---

% ---
% Configurações de aparência do PDF final

% alterando o aspecto da cor azul
\definecolor{blue}{RGB}{41,5,195}

% informações do PDF
\makeatletter
\hypersetup{
     	%pagebackref=true,
		pdftitle={\@title}, 
		pdfauthor={\@author},
    	pdfsubject={\imprimirpreambulo},
	    pdfcreator={LaTeX with abnTeX2},
		pdfkeywords={abnt}{latex}{abntex}{abntex2}{relatório técnico}, 
		colorlinks=true,       		% false: boxed links; true: colored links
    	linkcolor=black,          	% color of internal links
    	citecolor=black,        		% color of links to bibliography
    	filecolor=magenta,      		% color of file links
		urlcolor=blue,
		bookmarksdepth=4
}
\makeatother
% --- 

% --- 
% Espaçamentos entre linhas e parágrafos 
% --- 

% O tamanho do parágrafo é dado por:
\setlength{\parindent}{1.3cm}

% Controle do espaçamento entre um parágrafo e outro:
\setlength{\parskip}{0.2cm}  % tente também \onelineskip

% ---
% compila o indice
% ---
\makeindex
% ---

% ----
% Início do documento
% ----
\begin{document}

% Seleciona o idioma do documento (conforme pacotes do babel)
%\selectlanguage{english}
\selectlanguage{brazil}

% Retira espaço extra obsoleto entre as frases.
\frenchspacing 

% ----------------------------------------------------------
% ELEMENTOS PRÉ-TEXTUAIS
% ----------------------------------------------------------
% \pretextual

% ---
% Capa
% ---
\imprimircapa
% ---

% ---
% Folha de rosto
% (o * indica que haverá a ficha bibliográfica)
% ---
\imprimirfolhaderosto*
% ---

% ---
% LISTA DE ILUSTRAÇÕES
% ---
%\pdfbookmark[0]{\listfigurename}{lof}
%\listoffigures*
%\cleardoublepage
% ---

% ---
% LISTA DE TABELAS
% ---
%\pdfbookmark[0]{\listtablename}{lot}
%\listoftables*
%\cleardoublepage
% ---

% ---
% inserir o sumario
% ---
\pdfbookmark[0]{\contentsname}{toc}
\tableofcontents*
\cleardoublepage
% ---


% ----------------------------------------------------------
% ELEMENTOS TEXTUAIS
% ----------------------------------------------------------
\textual
% ---
% Introdução
% ---
\chapter{INTRODUÇÃO}

%%Adicione aqui um parágrafo introdutório para que o capítulo não comece ``vazio'' e a leitura não entre diretamente na primeira seção. Você pode começar o parágrafo falando do assunto e dando continuidade nas seções seguintes ou começar com um parágrafo genérico que descreve o que os leitores encontrarão. Exemplo:

À primeira vista com a início da exploração do lítio em 1991, pela Companhia Brasileira de Lítio, não trouxe tanto impacto para a população do vale do Jequitinhonha(Araçuaí), sobretudo a partir de 2023 com o início das operações pela Sigma Lithium e de outras mineradoras, há uma mudança nesse cenário. \cite{preocupacao-exploracao-litio}



%\textit{``Esse capítulo apresentará...''}

\section{CONTEXTUALIZAÇÃO}

%%Descreva o problema ou desafio que o sistema desenvolvido aborda. Explique por que esse problema é importante e qual será a efetividade do sistema. 

%%Busque fontes confiáveis para embasar seu argumentos.
 A partir dessa nova realidade que o lítio trouxe, acontece uma migração excessiva para a cidade de Araçuaí, e as circunvizinhas, em um tempo muito curto levando a uma mudança de rotinas de seus moradores, de agora em diante estaria pondo em prática a lei da oferta e da procura. 

\section{OBJETIVOS}
%Apresente claramente os objetivos do sistema, ou seja, o que se espera alcançar com sua implementação. 

%Normalmente, objetivos costumam ser divididos entre o principal (ou geral) e os específicos. O objetivo principal é a grande meta final que descreve qual resultados final deseja-se obter. Os objetivos específicos podem ser estruturados como metas secundárias que devem ser cumpridas para que o objetivo principal seja alcançado.
Objetivo Principal: Desenvolver um sistema web que simplifique e auxilie a locação de Imóveis, atendendo às necessidades de locadores e locatários na região impactada pela mineração e alta demanda imobiliária.



\section{PÚBLICO-ALVO E BENEFÍCIOS}
%Identifique e descreva o público-alvo do sistema, ou seja, o grupo de pessoas que se beneficiará com a sua solução (pense no público-alvo como os eventuais clientes de seu sistema). Aproveite para apresentar os eventuais benefícios que sua proposta trará para o público-alvo delimitado (explicar para um possível cliente o que será obtido com seu sistema ajuda a justificar porque ele(a) deveria investir no seu produto).
Público Alvo: Migrantes e Locatários em busca de imóveis na região impactada.  Proprietários e administradores de imóveis que desejem alugar suas propriedades.

Benefícios: 
Para Locatários: Acesso a uma plataforma centralizada para buscar, filtrar e selecionar propriedades de acordo com suas necessidades e desejos, com informações detalhadas e avaliações de outros usuários.

Para Locadores: Maior visibilidade para suas construções além de um sistema eficiente para gerenciar ofertas e comunicação com potenciais inquilinos.


%\section{Plano de monetização}

%Descreva como a equipe planeja monetizar o produto elaborado (sistema construído). Quais os eventuais valores, formas de cobrança e itens relacionados ao negócio. Para isso, suponha o 

%Busque fontes confiáveis para embasar seu argumentos.

% ---
% Atividades desenvolvidas
% ---
\chapter{Atividades desenvolvidas}

Adicione aqui um parágrafo introdutório para que o capítulo não comece ``vazio'' e a leitura não entre diretamente na primeira seção. Você pode começar o parágrafo falando do assunto e dando continuidade nas seções seguintes ou começar com um parágrafo genérico que descreve o que os leitores encontrarão. Exemplo:

\textit{``Esse capítulo apresentará...''}

\section{Escopo do Projeto}
Defina os limites do projeto, ou seja, o que está incluído no escopo e o que está excluído. Isso ajuda a evitar mal-entendidos futuros.

Liste e explique as funcionalidades e recursos do sistema. Isso pode incluir ações que os usuários podem realizar, como login, busca, envio de informações etc. Apresente aqui o Levantamento de Requisitos contendo os Requisitos Funcionais (RF) e os Não-Funcionais (RNF);

\section{Banco de Dados}
Explique como os dados são armazenados e gerenciados pelo sistema, incluindo detalhes sobre o banco de dados utilizado, esquema de banco de dados, tabelas e relacionamentos. Apresente aqui o Diagrama Entidade-Relacionamento (DER) e o Modelo Entidade-Relacionamento (MER);

\section{Arquitetura do Sistema}
Descreva a arquitetura geral do sistema. Explique como está a estruturação dos arquivos, como os códigos estão interligados e de que forma as funções estão associadas.

Caso deseje, separe a explicação desse tópico em front-end e back-end.

%\subsection{Front-end}
%\subsection{Back-end}

\section{Tecnologias e Ferramentas}
Mencione as tecnologias, linguagens de programação, \textit{frameworks}, bibliotecas (\textit{libs}) e demais ferramentas que foram utilizadas no desenvolvimento do sistema. Apresente aqui imagens e/ou logotipos das ferramentas.


% ---
% Conclusão
% ---
\chapter{CONSIDERAÇÕES FINAIS}
% ---

Com base nas pesquisas e conversas com alguns trabalhadores que residem em cidades vizinhas, e precisam se deslocarem para a cidade onde ficam localizados o porto de mineração e sedes da SIGMA Lítio, prestadoras terceirizadas para a mesma, tendo uma dificuldade para encontrar pontos comerciais e residenciais para residirem, Onde detectamos a necessidade de realizarmos em esquipe um sistema que todos tenham a facilidade de encontrar casas e pontos comerciais tanto para locação quanto para locatários independentes auxiliando juntamente com: Contratos Redigidos e orientados por um profissional na área de Direitos Imobiliários, sendo encontrados nos seguintes Anexos A e B. Todo contrato de aluguel deve seguir as leis vigentes na área, respeitando e seguindo a Constituição Brasileira. Para embasar nosso aplicativo, consultamos o advogado Dr. David Torres Silva, que redigiu dois modelos de contrato de aluguel.

O tema do trabalho era o desenvolvimento de uma aplicação web para a alocação de imóveis, a fim de facilitar a comunicação entre locador e locatário na cidade de Araçuaí e região, em que fosse possível fazer o cadastro de imóveis, agendamento de visitas, um sistema de chats e um modelo de contrato para que pessoas com pouco acesso a advogados soubessem como funciona um contrato de locação. Houve muitos marcos positivos, como o aprendizado de novas tecnologias, e pontos negativos, como a falta de suporte a bibliotecas antigas em novas versões do Laravel.\\

Acreditamos que o sistema tem muito a evoluir; as evoluções propostas são a criação de um sistema de avaliação, a criação de um sistema de casas favoritas e um sistema de denúncia de usuários/casas.









% ----------------------------------------------------------
% ELEMENTOS PÓS-TEXTUAIS
% ----------------------------------------------------------
\postextual

% ----------------------------------------------------------
% Referências bibliográficas
% ----------------------------------------------------------
\bibliography{referencias}


% ----------------------------------------------------------
% Anexos
% ----------------------------------------------------------

% ---
% Inicia os anexos
% ---
\begin{anexosenv}

% ---
\chapter{Título do anexo A}
% ---

Coloque o conteúdo do seu anexo aqui (se houver).


% ---
\chapter{Título do anexo B}
% ---

Coloque o conteúdo do seu anexo aqui (se houver).

% ---
\chapter{Título do anexo C}
% ---

Coloque o conteúdo do seu anexo aqui (se houver).




\end{anexosenv}

\printindex


\end{document}
