\begin{anexosenv}

% ---
\chapter{CONTRATO DE LOCAÇÃO DE IMÓVEL RESIDENCIAL}
% ---

CONTRATO DE LOCAÇÃO DE IMÓVEL RESIDENCIAL

LOCADOR: [Nome completo do locador], [nacionalidade], [estado civil], [profissão], portador do RG nº [número], inscrito no CPF/MF sob o nº [número], residente e domiciliado à [endereço completo];

LOCATÁRIO: [Nome completo do locatário], [nacionalidade], [estado civil], [profissão], portador do RG nº [número], inscrito no CPF/MF sob o nº [número], residente e domiciliado à [endereço completo];

IMÓVEL: O imóvel objeto da locação é situado à [endereço completo do imóvel], registrado sob matrícula nº [número] do Cartório de Registro de Imóveis de [cidade/estado];

FINALIDADE DA LOCAÇÃO: A locação é realizada para fins exclusivamente residenciais.

As partes acima identificadas têm, entre si, justo e acertado o presente Contrato de Locação de Imóvel Residencial, que se regerá pelas cláusulas e condições seguintes, bem como pelas disposições da Lei nº 8.245, de 18 de outubro de 1991 (Lei do Inquilinato) e demais legislações aplicáveis.

CLÁUSULAS E CONDIÇÕES

CLÁUSULA 1ª - DO OBJETO

O presente contrato tem por objeto a locação do imóvel descrito acima, de propriedade do LOCADOR, destinado à residência do LOCATÁRIO.

CLÁUSULA 2ª - DO PRAZO

A locação terá o prazo de [prazo do contrato] meses, com início em [data de início] e término em [data de término], podendo ser renovado mediante acordo prévio entre as partes.

CLÁUSULA 3ª - DO ALUGUEL

O valor mensal do aluguel será de RS valor do alugue, a ser pago até o dia data do vencimento de cada mês, na conta bancária indicada pelo LOCADOR ou em mãos, mediante recibo.
CLÁUSULA 4ª - DO REAJUSTE

O valor do aluguel será reajustado anualmente, de acordo com a variação do índice [especificar o índice de reajuste, como o IGP-M], ou outro índice oficial que venha a substituí-lo.
CLÁUSULA 5ª - DOS ENCARGOS DO LOCATÁRIO

O LOCATÁRIO fica responsável pelo pagamento das seguintes despesas durante a vigência do contrato:
- Contas de água, energia elétrica, gás e outras utilidades;
- Taxas de condomínio;
- IPTU proporcional ao período de locação, quando aplicável;
- Pequenos reparos no imóvel decorrentes do uso regular do mesmo.

CLÁUSULA 6ª - DOS DEVERES DO LOCADOR

O LOCADOR se compromete a:
- Entregar o imóvel em condições de uso e habitabilidade;
- Realizar reparos estruturais necessários, que não decorram do uso normal do imóvel pelo LOCATÁRIO;
- Respeitar o direito de posse do LOCATÁRIO enquanto durar o contrato, salvo os casos previstos em lei.

CLÁUSULA 7ª - DOS DEVERES DO LOCATÁRIO

O LOCATÁRIO se compromete a:
- Utilizar o imóvel conforme sua destinação (uso residencial);
- Zelar pela boa conservação do imóvel, comunicando ao LOCADOR a necessidade de reparos urgentes;
- Não realizar reformas ou modificações sem a prévia autorização por escrito do LOCADOR;
- Restituir o imóvel nas mesmas condições em que o recebeu, salvo o desgaste natural do uso regular.

CLÁUSULA 8ª - DA MULTA POR DESCUMPRIMENTO

Em caso de infração de qualquer cláusula deste contrato, a parte infratora pagará à parte prejudicada uma multa equivalente a [valor da multa ou percentual sobre o aluguel], sem prejuízo da indenização por eventuais danos.

CLÁUSULA 9ª - DA GARANTIA LOCATÍCIA

Como garantia do presente contrato, o LOCATÁRIO apresenta [especificar a modalidade de garantia: fiador, caução, seguro-fiança ou título de capitalização], nos termos do art. 37 da Lei nº 8.245/91.

CLÁUSULA 10ª - DA RESCISÃO

O presente contrato poderá ser rescindido:
- Pelo LOCATÁRIO, mediante notificação com [prazo de aviso prévio, geralmente 30 dias];
- Pelo LOCADOR, em caso de inadimplemento por parte do LOCATÁRIO ou nas hipóteses previstas no art. 9º da Lei nº 8.245/91;
- Pelo comum acordo entre as partes.

CLÁUSULA 11ª - DAS BENFEITORIAS

As benfeitorias realizadas pelo LOCATÁRIO, ainda que necessárias, não serão indenizáveis, salvo acordo escrito em contrário. As úteis e voluptuárias somente poderão ser realizadas mediante prévia autorização do LOCADOR.

CLÁUSULA 12ª - DO DIREITO DE PREFERÊNCIA

O LOCATÁRIO terá o direito de preferência na compra do imóvel, nas mesmas condições oferecidas a terceiros, conforme o art. 27 da Lei nº 8.245/91, devendo manifestar seu interesse no prazo de 30 dias após notificação do LOCADOR.

CLÁUSULA 13ª - DAS DISPOSIÇÕES FINAIS

Qualquer alteração neste contrato deverá ser feita por escrito e assinada por ambas as partes.

CLÁUSULA 14ª - DO FORO

Fica eleito o foro da comarca de [cidade/estado], para dirimir quaisquer dúvidas ou litígios oriundos deste contrato, com renúncia a qualquer outro, por mais privilegiado que seja.

E por estarem assim justos e acordados, assinam o presente instrumento em 2 (duas) vias de igual teor e forma, na presença das testemunhas abaixo.



\begin{center}
	[Local], [Data] \par
	\vspace{0.75cm}
	\hrulefill \par
	
	\textbf{[Assinatura do LOCADOR]} \par
	\textbf{[Nome do LOCADOR]} \par
	\textbf{[CPF do LOCADOR]} \par
	\vspace{0.75cm}
	\hrulefill \par
	
	\textbf{[Assinatura do LOCATÁRIO]} \par
	\textbf{[Nome do LOCATÁRIO]} \par
	\textbf{[CPF do LOCATÁRIO]} \par
	\vspace{0.75cm}
	\textbf{TESTEMUNHAS:} \par
	\vspace{1cm}
	
	1. \hrulefill \par
	\textbf{[Nome, assinatura e CPF da testemunha]} \par
	\vspace{1cm}
	
	2. \hrulefill \par
	\textbf{[Nome, assinatura e CPF da testemunha]} \par
\end{center}







% ---
\chapter{CONTRATO DE LOCAÇÃO DE IMÓVEL NÃO RESIDENCIAL}
% ---

CONTRATO DE LOCAÇÃO DE IMÓVEL NÃO RESIDENCIAL

LOCADOR: [Nome completo do locador], [nacionalidade], [estado civil], [profissão], portador do RG nº [número], inscrito no CPF/MF sob o nº [número], residente e domiciliado à [endereço completo];

LOCATÁRIO: [Nome completo do locatário], [nacionalidade], [estado civil], [profissão], portador do RG nº [número], inscrito no CPF/MF sob o nº [número], residente e domiciliado à [endereço completo];

IMÓVEL: O imóvel objeto da locação é situado à [endereço completo do imóvel], registrado sob matrícula nº [número] do Cartório de Registro de Imóveis de [cidade/estado];

FINALIDADE DA LOCAÇÃO: A locação é realizada para fins exclusivamente comerciais.

As partes acima identificadas têm, entre si, justo e acertado o presente Contrato de Locação de Imóvel Não Residencial, que se regerá pelas cláusulas e condições seguintes, bem como pelas disposições da Lei nº 8.245, de 18 de outubro de 1991 (Lei do Inquilinato) e demais legislações aplicáveis.

CLÁUSULAS E CONDIÇÕES

CLÁUSULA 1ª - DO OBJETO

O presente contrato tem por objeto a locação do imóvel descrito acima, de propriedade do LOCADOR, destinado ao uso comercial do LOCATÁRIO.

CLÁUSULA 2ª - DO PRAZO

A locação terá o prazo de [prazo do contrato] meses, com início em [data de início] e término em [data de término], podendo ser renovado mediante acordo prévio entre as partes, nos termos da Lei nº 8.245/91, art. 51 e seguintes, que tratam da renovação de locações não residenciais.

CLÁUSULA 3ª - DO ALUGUEL

O valor mensal do aluguel será de RS [valor do aluguel], a ser pago até o dia [data do vencimento] de cada mês, na conta bancária indicada pelo LOCADOR ou em mãos, mediante recibo.


CLÁUSULA 4ª - DO REAJUSTE

O valor do aluguel será reajustado anualmente, conforme a variação do índice [especificar o índice de reajuste, como o IGP-M], ou outro índice oficial que venha a substituí-lo.

CLÁUSULA 5ª - DOS ENCARGOS DO LOCATÁRIO

O LOCATÁRIO será responsável pelo pagamento das seguintes despesas durante a vigência do contrato:
- Contas de água, energia elétrica, gás e outras utilidades;
- Taxas de condomínio (se houver);
- IPTU proporcional ao período de locação;
- Pequenos reparos e manutenção do imóvel decorrentes do uso regular.

CLÁUSULA 6ª - DOS DEVERES DO LOCADOR

O LOCADOR se compromete a:
- Entregar o imóvel em condições de uso, adequado ao exercício da atividade comercial do LOCATÁRIO;
- Realizar reparos estruturais necessários, que não decorram do uso regular pelo LOCATÁRIO;
- Garantir ao LOCATÁRIO o uso pacífico do imóvel durante a vigência do contrato.

CLÁUSULA 7ª - DOS DEVERES DO LOCATÁRIO

O LOCATÁRIO se compromete a:
- Utilizar o imóvel conforme sua destinação (uso comercial);
- Zelar pela boa conservação do imóvel, comunicando ao LOCADOR qualquer necessidade de reparos estruturais;
- Não realizar reformas ou modificações sem a prévia autorização por escrito do LOCADOR;
- Restituir o imóvel nas mesmas condições em que o recebeu, salvo o desgaste natural do uso regular;
- Manter em dia o pagamento de tributos e encargos incidentes sobre o imóvel.

CLÁUSULA 8ª - DA MULTA POR DESCUMPRIMENTO

Em caso de descumprimento de qualquer cláusula deste contrato, a parte infratora pagará à parte prejudicada uma multa equivalente a [valor da multa ou percentual sobre o aluguel], sem prejuízo da reparação por eventuais danos.


CLÁUSULA 9ª - DA GARANTIA LOCATÍCIA

Como garantia do presente contrato, o LOCATÁRIO apresenta [especificar a modalidade de garantia: fiador, caução, seguro-fiança ou título de capitalização], conforme o art. 37 da Lei nº 8.245/91.

CLÁUSULA 10ª - DA RESCISÃO

O presente contrato poderá ser rescindido:
- Pelo LOCATÁRIO, mediante notificação com [prazo de aviso prévio, geralmente 30 dias];
- Pelo LOCADOR, em caso de inadimplemento por parte do LOCATÁRIO ou nas hipóteses previstas no art. 9º da Lei nº 8.245/91;
- Pelo comum acordo entre as partes.

CLÁUSULA 11ª - DAS BENFEITORIAS

As benfeitorias necessárias realizadas pelo LOCATÁRIO não serão indenizáveis, salvo acordo contrário entre as partes. As benfeitorias úteis e voluptuárias somente poderão ser realizadas com autorização prévia e por escrito do LOCADOR, podendo ou não ser indenizadas ao final da locação.

CLÁUSULA 12ª - DO DIREITO DE PREFERÊNCIA

O LOCATÁRIO terá o direito de preferência na compra do imóvel, caso o LOCADOR deseje vendê-lo, conforme previsto no art. 27 da Lei nº 8.245/91. O LOCATÁRIO terá o prazo de 30 dias, contados do recebimento da notificação, para manifestar interesse em adquirir o imóvel nas mesmas condições oferecidas a terceiros.

CLÁUSULA 13ª - DA RENOVAÇÃO DA LOCAÇÃO

O LOCATÁRIO, nos termos do art. 51 da Lei do Inquilinato, poderá pleitear a renovação compulsória da locação, desde que preencha os requisitos legais, tais como:
- Exercer atividade comercial há mais de três anos no imóvel;
- Ter contrato escrito e com prazo mínimo de cinco anos.

CLÁUSULA 14ª - DA TRANSFERÊNCIA OU CESSÃO

O LOCATÁRIO não poderá transferir ou ceder a locação, no todo ou em parte, sem o consentimento expresso e por escrito do LOCADOR.



CLÁUSULA 15ª - DAS DISPOSIÇÕES FINAIS

Qualquer alteração neste contrato deverá ser feita por escrito e assinada por ambas as partes.

CLÁUSULA 16ª - DO FORO

Fica eleito o foro da comarca de [cidade/estado], com renúncia a qualquer outro, por mais privilegiado que seja, para dirimir quaisquer dúvidas ou litígios oriundos deste contrato.

E por estarem assim justos e contratados, assinam o presente contrato em 2 (duas) vias de igual teor e forma, na presença das testemunhas abaixo.

\begin{center}
	[Local], [Data] \par
	\vspace{0.75cm}
	\hrulefill \par
	
	\textbf{[Assinatura do LOCADOR]} \par
	\textbf{[Nome do LOCADOR]} \par
	\textbf{[CPF do LOCADOR]} \par
	\vspace{2cm}
	
	\hrulefill \par
	
	\textbf{[Assinatura do LOCATÁRIO]} \par
	\textbf{[Nome do LOCATÁRIO]} \par
	\textbf{[CPF do LOCATÁRIO]} \par
	\vspace{0.75cm}
	\textbf{TESTEMUNHAS:} \par
	\vspace{1cm}
	
	1. \hrulefill \par
	\textbf{[Nome, assinatura e CPF da testemunha]} \par
	\vspace{1cm}
	
	2. \hrulefill \par
	\textbf{[Nome, assinatura e CPF da testemunha]} \par
\end{center}
\end{anexosenv}

